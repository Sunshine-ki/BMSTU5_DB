% Глава.	
\chapter{Лекция первая. Введение в БД.}

\section{Базы данных.}

\textit{БД} - это самодокументированная собрание интегрированных записей. Набор таблиц.

\textit{Самодокументированная} - хранятся метаданные, т.е. данные о данных.

Интегрированные записи - файлы данных. Целый комплекс.

\section{Основные требования к БД.}
\begin{itemize}
	\item Не избыточность - не храним лишнюю информацию.
	\item Эффективность доступа - малое время отклика на действие пользователя.
	\item Совместное использование.
	\item Безопасность. Также внутренняя безопасность - защита от дурака (пример: вместо числа ввел букву).
	\item Восстановление после сбоя.
	\item Целостность - если ссылаемся на какой-то объект, то он должен быть. Не ссылаться на несуществеющие объекты.
	\item Независимость от сторонних приложений. Если программа отправляет ерунду БД должна обработать.
\end{itemize}

\section{СУБД, журнализация.}

\textit{СУБД} - (Средства управления БД) приложение, обеспечивающее создание, хранение, обновление
и поиск информации в БД.Программа.
\textbf{СУБД управляет БД}.

\textit{Система БД} - совокупность БД.

\textit{Транзакция} - набор действий, которые выполняются одновременно.
(Пример: онлайн перевод, одновременно в одном месте деньги ушли, в другом появились.)

\textit{Журнализация} - информация о действиях, которые происходили в системе. Помогает
в откате каких-то действий. \textbf{БД} сохраняет запросы в журнале.

\textbf{СУБД должна поддерживать языки.}

\section{Основные компоненты СУБД}

\begin{itemize}
	\item Ядро - управление памятью. Журнализация.
	\item Процессор языка БД - оптимизация. Выполнение.
	\item Подсистема поддержки времени исполнения.
	\item Сервисные программы - те утилиты, которые мы пишем, доп. возможность. (Вывод звездочек вокруг имени.)
\end{itemize}

\section{Классификация СУБД}

\begin{itemize}
	\item По модели данных
	      \begin{itemize}
		      \item Дореляционная.
		            \begin{itemize}
			            \item Инвертированный список (рис 1)
			            \item Иерархия. (Дерево)
			            \item Сетевые (граф)
		            \end{itemize}
		      \item Реляционная.
		      \item Постреляционная
	      \end{itemize}
	\item По архитектуре.
	      \begin{itemize}
		      \item Локальные - на одном устройстве.
		      \item Распространенные - на многих устройствах.
	      \end{itemize}
	\item По способу доступа к БД
	      \begin{itemize}
		      \item Файл-серверный подход - Подключились, взяли всё. Нагружаем клиента,
		            а не сервер. \textbf{Минусы: У каждого клиента своя копия.}
		      \item клиент-серверные - запросы выполняются на сервере, клиент получает
		            только нужное
		      \item Встраиваемые - маленькие базы, которые не нужны всем.
	      \end{itemize}
\end{itemize}

