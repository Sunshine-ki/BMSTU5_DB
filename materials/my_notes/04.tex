\chapter{Функциональная зависимость (Лекция 5)}

\textbf{Функциональная зависимость (ФЗ)}

R - отношение

x, y - подмножество мн-ва атрибутов.

x->y <=> любое x связано в точности с одним y (Биекция)

\{Sno\} -> \{City\}

детерминант - левая часть.

зависимая часть - правая.

Функциональную зависимость строим исходя из имеющихся данных.
А не на основе какой-то логике.

Зависимая часть может содержать несколько значений.
Детерминант тоже может содержать нес. значений.

\textbf{Тривиальная функциональная зависимость} - когда y явл. подмножеством x.

\{Sno, Pno\} -> \{Sno\}

\{Sno, Qty\} -> \{Qty\}

Множество всех функциональных зависимостей, которые задаются данным множеством
ФЗ является \textbf{замыканием множества}.

\textbf{Правило Амстронга}

\begin{itemize}
	\item Правило рефлексивности: Если B подможество A, то B функционально зависит от А (A->B)
	\item Правило дополнения: Если B функционально зависит от А (A->B), то АС->BC
	      (Т.е. мы можем добавить абрибут справа и слева)
	\item Правило транзитивности: A->B, B->C => A->C
	\item (Выше были основные далее вытекают из ранее приведенных)
	      Самоопределния: A->A
	\item Декомпозиции: A->BC => A->B и A->C
	\item Объединения: A->B и  A->C => A->BC
	\item Композиция: A->B и C-> D => AC -> BD
	\item Общая теорема объединения: A->B и C->D => A(C-B)->BD
\end{itemize}

Пример (Задач, которые будут на РК)

1. R(A,B,C,D,E,F)

S = \{
A->BC,

B->E,

CD->EF
\}

Задача: AD->F ?

Решение:

1. A->BC => A->B и A->C

2. A->C => AD->CD

3, AD->CD И CD->EF => AD->EF

4. AD->EF => AD->F

\textbf{Супер ключ} - Супер ключ R - множество атрибутов R,
котор. содержит в виде подмножества хотя бы один
(не обязательно собственный) потенциальный ключ.
(ключ с множество ключей с доп. атрибутами).

К - подмножество R

K->A для любого A принадлежащего R.

\textbf{Алгоритм нахождения ключа}

1. K = R

2. Для каждого атрибута из K выполняем след. действия

2.1 Вычислим замыкание {K-A}+

2.2 Если замыкание {K-A}+ = R, то K = {K-A}+

\textbf{Алгоритм вычисления замыкания}

\begin{enumerate}
	\item $J_(new)$ = k
	\item repeat
	\item $J_(old)$ = $J_(new)$
	\item foreach (X->Y in S) do
	\item J(X подмножество $J_(new)$) then $J_(new)$ = $J_(new)$ + J
	\item until ($J_(old)$ = $J_(new)$)
\end{enumerate}

Пример:

R(A,B,C,D,E,F)
S = \{
A->BC,

E->CF,

B->E,

CD->EF
\}

Найти: {A, B}+ ?


Решение: {A, B}+ = {A, B, C, E, F}

ЕСЛИ БЫ ТАМ ЕЩЕ БЫЛО D, ТО ЭТО ЯВЛЯЛОСЬ БЫ ПОТЕНЦИАЛЬНЫМ КЛЮЧОМ.

Два множества ФЗ S1 и S2 явл \textbf{эквивалентными} тогда и только тогда
когда они явл. покрытиями друг друга.

Пример:

Есть F - Набор ФЗ.

F = \{
A -> C,

AC -> D,

E -> AD,

E -> H
\}

G = \{
A -> CD,

E -> AH
\}

Задача: Доказать что они явлю эквивалентными (или не явл.)

Решение:

1. G - покрывает F ?

\{A\}+ = \{A,C,D\} (Строим по F) = A->CD (По G)

\{E\}+ = \{E, A, D, H, C\} (ПО F) = \{E, A, H, C, D\} (Множества совпадают, значит G покрывает F)

1. F - покрывает G ?

(По кому покрываем по тому и строим)

\{A\}+ = \{A,C,D\} (По G) = \{A,C,D\}

\{AС\}+ = \{A,C,D\} (По G) = \{ACD\}

\{E\}+ = \{E, A, D, H, C\} (По G) = \{E, A, H, C, D\} => F эквивалентно G.

Множество ФЗ явл. \textbf{неприводимым} тогда и только тогда, когда обладает след. свойствами:

\begin{enumerate}
	\item Любая ФЗ X->Y, Y - один элемент
	\item Ни одну ФЗ нельзя удалить без изменения замыкания
	\item Ни один атрибут не может быть удален из детерминента без изменения замыкания
\end{enumerate}

Примеры (Явл. min покрытием?):

1. S:

\{Pno\} -> \{Pname, Color\}

\{Pno\} -> \{City\}

Нарушает! =>

\{Pno\} -> \{Pname\}

\{Pno\} -> \{Color\}

2.

S:

Pno -> Pno

Pname -> Pname

Pno -> City

Нарушает! =>

Pno -> City

3. E - множество ФЗ.

min покрытие F (Проверяем все правила)

1. F = E

X -> \{A1,...,An\} =>

декомпозиция:

X - > A1

X - > An

2. Любая ФЗ X -> A из F

Любой B из X:

F - \{X-A\} объединение \{(X-B) -> A\} = F

3. Для каждой ФЗ, где A -> X

Мы проверяем

F - \{X -> A\} = F

Если так, то мы можем удалить.

Пример (Найти min покрытие):

R (A,B,C,D)

A -> BC

B -> C

A -> B

AB -> C

AC -> D

1. Разбиваем (Удаляем те ФЗ, которые выводимы)

A -> B

A -> C (Это можно удалить)

B -> C

A -> B (Это тоже можно удалить )

AB -> C

AC -> D

2. Разбиваем (Удаляем те ФЗ, которые выводимы)

Объединяем первые две и делаем композицию с третьей и получается выводим третью,
значит ее можно удалить

A -> B

B -> C

AB -> C

AC -> D

3.

A -> B

B -> C

AC -> D

4.

A -> B

B -> C

A -> D


\chapter{Семинар 4}

Процедуры и функции.

Параметры - имя, тип.

Отличие: параметры у процедуры не берутся в скобки.

@ - локальная переменная.

@@ - глобальная переменная.

Параметры которые передаются по ссылки передаются с OUTPUT

Глубина рекурсии - 32.

create procedure [схема].имя  <параметры>

[with <опции>]

as

<тело функции>

end;

Пример: (Факториал)

У postgres'a другие ключевые слова!

\begin{lstlisting}[label=some-code,caption=Пример]
create procedure dbo.Factorial @Valin bigint, 
@ValOut bigint OUTPUT
as
begin
if @Valin > 20
begin
print N'Error'
return -99
end
-- 'Объявляем переменную'
declare @WorkValin bigint
-- 'Создаем переменную'
@WorkValout bigint
if @Valint != 1
begin
set @WarkValin = @Valin - 1
print @@NestedLevel
exec dbo.Factorial @WarkValIn, @WorkValout
set @ValOut = @WorkIN(WorkValOut)
end;
else
set @ValOut
end
\end{lstlisting}

Вызов процедуры:

\begin{lstlisting}[label=some-code,caption=Пример]
-- 'Определяем две переменные'
declare @FactIn
bigint
declare @Factout
bigint
set dbo.Factorial @Factin @FactOut OUTPUT
print convert(Varchar(20) @Factout)
\end{lstlisting}

Пишем истинную процедуру, которая ничего не возвращает, а лишь изменяет таблицу.

Имеем таблицу.


\begin{table}[ht]
	\centering
	\begin{tabular}{ | l | l ||}
		\hline
		id  & name  \\ \hline
		1   & test1 \\ \hline
		2   & test2 \\ \hline
		3   & test3 \\ \hline
		... & ...   \\ \hline
		n   & testn \\ \hline
		\hline
	\end{tabular}
	\caption{Таблица}
\end{table}


Задание: Добавить данные в таблицу.
Добавить в конец еще одну строку.

Можно вставить python код.

\$\$ - тело процедуры в долларах

(В первую переменную ( maxid ) запишется max(id) + 1)
(Во вторую ( maxname ) 'test' || max(id) + 1)

\begin{lstlisting}[label=some-code,caption=Пример]
create function addTest()
returns void
language PipgSQL
as $$ 
declare 
	maxid int
	maxname Varchar(10)
begin
	select max(id) + 1, 'test' || max(id) + 1
	from test
	into maxid, maxname 
	insert into test(id, name)
	values (maxis, maxName)
\end{lstlisting}

\textbf{Либо вот так:}

\begin{lstlisting}[label=some-code,caption=Пример]
create function addTest()
returns void
language PipgSQL
as $$ 
declare 
	maxid int
	maxname Varchar(10)
begin
	select insert into test
	values ('||max(id)+1|', "test'||max(id)+1||'");
	from test
	into maxid, maxname 
\end{lstlisting}

\textbf{Запускаем:}

query execute qSrting

\textbf{Триггер} - Объект. Ответ на событие. Реакция.

\textbf{Виды:}

\begin{enumerate}
	\item DDl -триггеры (create, drop, alter)
	\item DDM - триггер (insert, delete, update)
	      \begin{enumerate}
		      \item instead of (Вместо) (Кол-во 1)
		      \item before/alter Реагируют на какое-то действие и добавляют свое (делают действие до и после) (Вместе с) (Кол-во бесконечено)
	      \end{enumerate}
	      \begin{table}[ht]
		      \centering
		      \begin{tabular}{ | l | l | l|}
			      \hline
			      Хар-ка     & instead of             & before/alter \\ \hline
			      тип        & вместо                 & вместе с     \\ \hline
			      кол-во     & 1                      & бесконечно   \\ \hline
			      применение & таблица, представления & таблица      \\ \hline
			      \hline
		      \end{tabular}
		      \caption{Таблица}
	      \end{table}
\end{enumerate}

for - указываем на какие события мы реагируем.

as - после этого ключевого слова указывает действия, которые хотим сделать.

\begin{lstlisting}[label=some-code,caption=Пример DDL триггера.]
create trigger safety 
on database
for drop_table, alter_table
as begin
	print N'Error'
	rollback 
end;
\end{lstlisting}

Создадим триггер.

\begin{lstlisting}[label=some-code,caption=Создадим триггер.]
create trigger inserSP
on SP
after update
as 
begin
	raiserror 'Новая подставка'
end;
\end{lstlisting}

\section{Курсор}

Курсор - это набор из результата sql запроса и указатель. (Зло). Это объект.

Используется - когда нужно выполнить что-то в цикле (к примеру удалть объекты)

Классификация

\begin{enumerate}
	\item По области видимости:
	      \begin{itemize}
		      \item Локальные
		      \item Глобальные
	      \end{itemize}
	\item По типу:
	      \begin{itemize}
		      \item Static (Статичный) (Требует доп. ресурса. Также таблица не должна менять)
		      \item Dynamic (Динамический). Позволяет отлеживать все изменения. (На это нужно огромные ресурсы. Тем самым время)
		      \item Key set - модификация динамического курсора.
		      \item Fast forward - всегда идем только вниз (просматриваем вниз)
	      \end{itemize}
	\item По способу перемещения.
	      \begin{itemize}
		      \item Forward\_only Вниз
		      \item Scroll - Гуляем туда сюда.
		      \item По параллельному доступу.
		            \begin{enumerate}
			            \item read-only
			            \item Optimistic - разрешаем другим читать.
			            \item Scroll lock (Пессимистичная блокировка) - Не дает доступа другим.
		            \end{enumerate}

	      \end{itemize}
\end{enumerate}

\begin{lstlisting}[label=some-code,caption=Курсор]
declare 'имя курсора'
cursor 
['область видимости']
['тип']
['Способ применения']
['Параллельный доступ']
for <sql-запрос>
\end{lstlisting}

\begin{lstlisting}[label=some-code,caption=Курсор. Пример]
declare myCursor cursor
for select Sname, Pname
from S join SP
on S.Sno = SP.Sno
join P on P.Pno = Sp.Pno

declare @Swork Varchar(10), @Pwork Varchar(10)

open myCursor

-- fetch - 'Запись переменной'
-- next - 'В каком' 'порядке'
fetch next from myCursor 
into @Swork, @Pwork
-- - @@Fetch_status - 0 - 'больше не' 'можем читать' (EOF)
while @@Fetch_status == 0
begin fetch next from myCursor
into ...
end;
close myCursor
\end{lstlisting}

\section{Индексы}

Оптимизация запросов.

В основе лежит b\_tree - сбалансированное дерево.

Класторизованный индекс - Реальные данные в листе.

\begin{enumerate}
	\item Можно создать 1 класторизованный индекс
	\item primary key / unique
\end{enumerate}

Некласторизованный и ндекс - Если хранятся ссылки на объекты.

\begin{enumerate}
	\item n штук
	\item Создается по запросу
\end{enumerate}


\begin{lstlisting}[label=some-code,caption=Индекс]
select *
from test
where id = 29
\end{lstlisting}

\begin{lstlisting}[label=some-code,caption=Индекс. Создание]
create nonclustered index my myId
on SP
include (Sno, Pno)

select *
from SP
where Qty=5
\end{lstlisting}

\section{Партицирование}

 (partirion by)

Разбиение данных в таблице на подтаблицы.