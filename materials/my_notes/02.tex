% Глава.	
\chapter{Реляционная модель}

\section{ER - модель}

\begin{itemize}
	\item Сущность
	\item Связь
\end{itemize}

Объекты обозначаются прямоугольниками. Внутри пишем название.

\textbf{Виды сущностей:}
\begin{itemize}
	\item Сильные - Обозначаются просто в рамке.
	\item Слабые - не могут существовать друг без друга. Факультет и предметы.
	      Обозначается вложенным квадратом (рамочка).
\end{itemize}

\textbf{Атрибуты} отображаются овалами. Внутри пишем название атрибута.

\textbf{Виды связей:}
\begin{itemize}
	\item Один к одному. Студент-зачетка.
	\item Один ко многим. Статья-рецензия. Добавляем внешний ключ со стороны многих.
	      Из многих в сторону одного.
	\item Многие ко многим. Студент-преподаватель. Добавляем связочную таблицу.
\end{itemize}

\section{Реляционная модель}

\textbf{Реляционная модель}
\begin{itemize}
	\item Структурная часть - отвечает за то, какие объекты есть.
	\item Целостная - отвечает за ссылки. DDL.
	      \begin{itemize}
		      \item Ссылочная целостность (FK)
		      \item Целостность сущности (PK) - говорит о том, что есть первичный ключ.
		            Нет повторения. Всегда знаем на что ссылаемся.
	      \end{itemize}
	\item Манипуляционная - за механизм работы с данными. DML.
\end{itemize}

\textbf{Домен} = (примерно равно) тип данных.\\
\textbf{Атрибут} (отношения) = (примерно равно) столбец. Упорядоченная пара вида:\\
имя-атрибута,имя-домена\\
\textbf{Схема отношений} = (примерно равно) Заголовок. имя-отношение, имя-домена\\
\textbf{Кортеж} = (примерно равно) Строка. Имя-атрибута, значение-атрибута
\textbf{Отношение} = (примерно равно) таблица.

Непустое подмножество множества атрибутов схемы отношения будет \textbf{потенциальным ключом} тогда и только тогда,
когда оно будет обладать свойствами:
\begin{itemize}
	\item уникальности (в отношении нет двух различных кортежей с одинаковыми
	      значениями потенциального ключа)
	\item неизбыточности (никакое из собственных подмножеств множества
	      потенциального ключа не обладает свойством уникальности
\end{itemize}

\textbf{Внешний ключ} в отношении R2 – это непустое подмножество множества атрибутов FK этого отношения, такое, что:
\begin{itemize}
	\item Существует отношение R1 (причем отношения R1 и R2 необязательно различны) с потенциальным ключом CK;
	\item Каждое значение внешнего ключа FK в текущем значении отношения R2 обязательно совпадает со значением ключа CK
	      некоторого кортежа в текущем значении отношения R1.
\end{itemize}


