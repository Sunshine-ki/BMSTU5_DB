\chapter{Семинар 5}


\section{Грамматика}

\begin{itemize}
	\item Атом - зарезервированные слова (и переменные)
	\item Категории.
\end{itemize}

Построение грамматики:

\textbf{SFW} - Select From Where.

% Select <shell.slt>

\chapter{Теория проектирования }


Типы проектирования:

\begin{itemize}
	\item Логическое
	\item Физическое
\end{itemize}

Аномалии:

\begin{itemize}
	\item Вставки
	\item Обновления (Когда дублируется информация. В одном месте обновили в другом забыли)
	\item Удаления (Две слипшиеся сущности. Не можем по отдельности удалить.)
\end{itemize}


\section{Формализация}

Формализация - убирает аномалии.

\textit{Декомпозиция без потерь} - это

Пример:

R(a, b, c)

Разбиваем:

N = $2^i$ = $2^3$ = 8

1. \{a\},  \{b\},  \{c\}

2.  \{a\},  \{b, c\}

3.  \{a, b\},  \{c\}

\dots

\textbf{Без потери} - значит, что если объединим (сделаем JOIN) ничего не потеряем.

Нормальные формы:

\begin{itemize}
	\item 1NF - каждая след. норма форма лучше предыдущей. Каждый атрибут явл.
	      атомарным (нужно избавляться от массивов).
	\item 2NF - Должна быть первая норм форма (1NF)
	      и каждый неключевой атрибут функционально полно неприводимо зависит от потенциального ключа
	\item 3NF - это 2NF и каждый неключевой атрибут нетривиально зависит от первичного ключа.
	\item BCNF (Бойса-Кодда) - это 3NF с улучшениями (оптимизация).
	      Каждая нетривиальная и неприводимая слева функциональная зависимость
	      имеет в кач-ве детерминанта перевичный ключ.
	\item 4NF - это 3NF или BCNF и все нетривиальные многозначные зависимости функционально
	      зависят от  потенциальных ключей.
	\item 5NF - 4NF и нетривиальная зависимость соединения определяется потенциальным ключом.
	\item 6NF - находится в 5NF и больше ее декомпозицировать нельзя.
\end{itemize}

Денормализация - обратный процесс нормализации. Объединяем таблицы.